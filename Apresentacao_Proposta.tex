%%% Dichiarazione dei pacchetti standard.
\documentclass{beamer}
\usepackage[utf8]{inputenc}
\usepackage[brazilian]{babel}
\usepackage{graphicx}
\usepackage{wrapfig}
\usepackage{amsfonts}
%\usepackage[square, numbers, comma, sort&compress]{natbib}  % Use the
                                % "Natbib" style for the references in
                                % the Bibliography

%%% Personalizzazione del layout---articolata su cinque livelli.
%\usetheme{split}        % layout complessivo. 
%\usetheme{Goettingen}        % layout complessivo. 
%\usetheme{CambridgeUS}        % layout complessivo. 
%\usetheme{Boadilla}        % layout complessivo. 
%\usetheme{Montpellier}        % layout complessivo. 
%\usetheme{Warsaw}        % layout complessivo. 
%\usetheme{Pittsburgh}        % layout complessivo. 
\usetheme{Hannover}        % layout complessivo. 

\useinnertheme{default} % layout interno.
\useoutertheme{default} % layout esterno.
\usecolortheme{default} % schema di colori.
\usefonttheme{default}  % schema dei font.
% Inutile dire che se volete tutti i default, potete risparmiarvi gli ultimi
% quattro comandi. 
\definecolor{jgreen}{RGB}{0,92,40}
\definecolor{jyellow}{RGB}{211,168,0}
\definecolor{jred}{RGB}{153,32,0}
\beamertemplatenavigationsymbolsempty 
\setbeamercolor{sidebar}{bg=jgreen}
\setbeamercolor{section in sidebar}{fg=yellow}
\setbeamercolor{title in sidebar}{fg=white}
\setbeamercolor{author in sidebar}{fg=red}
\setbeamercolor{frametitle}{fg=jgreen}
\setbeamercolor{title}{fg=jgreen}
\setbeamercolor{item}{fg=jgreen}

%%% Titolo e autore.

\title{Gestao compartilhada dos recursos da Rede Mocambos}
%\subtitle{Architettura e prototipo per la Rete Mocambos}
\author{TC e Vince}
\institute{Casa de Cultura Tain\~{a} - Rede Mocambos}
\date{2 Maio 2012}

\begin{document}

% Local background must be enclosed by curly braces for grouping.
{
\usebackgroundtemplate{
  \hspace{0.65cm}
  \includegraphics[width=.45\textwidth]{./FIG/logomocambos.pdf}
}%
\begin{frame}
  \titlepage
\end{frame}


\section{Rede Mocambos}

%{
%\usebackgroundtemplate{
%  \hspace{2cm}
%  \begin{picture}(0,0)\put(200,-270)
%    {\includegraphics[width=0.5\textwidth]{./FIG/logomocambos.pdf}}
%  \end{picture}}

\begin{frame}
  \frametitle{a Rede Mocambos hoje}

  \begin{itemize}
    \item Mais de 200 comunidades (quilombos, terreiros, aldeias, \ldots)
    \item Conectividade e telecentros (GESAC, Telecentros.BR, \ldots)
    \item Articulação e coordenação de projetos das comunidades da
      Rede e dos NFCs (pajelanças, encontros nacionais e regionais, metarec,
      geradores, gravações, \ldots)
    \item Desenvolvimento e gestao da infraestrutura tecnologica
      (portal, wiki, email, mapa, servidores, \ldots)  
    \end{itemize}
\end{frame}

\begin{frame}
  \frametitle{Geoinformatica}
  Ferramentas informaticas com dados geofraficos.
  \begin{itemize}
    \item Comunidades em 23 estados
    \item Implementação e desenvolvimento da ``Rota dos Baobás'' -
      \url{http://mapa.mocambos.net} (Software Livre africano, Ushahidi)
    \item Cadastramento das comunidades e categorização por tipo e programa
    \item Etiquetas para situação/estado (tags)
    \end{itemize}
\end{frame}

\section{Rota dos Baob\'{a}s}

% {
% \usebackgroundtemplate{
% %  \hspace{2cm}
%   \begin{picture}(0,0)\put(130,-270)
%     {\includegraphics[width=0.8\textwidth]{./Figure/diversidade.pdf}}
%   \end{picture}}

% \begin{frame}

%   \frametitle{Brasile: Uno, nessuno e centomila}
%   Il Brasile è un paese di dimensione continentale dove resistono
%   molte culture in ambienti e contesti molto diversi. 
%   \begin{itemize}
%   \item Indios
%   \item Quilombola
%   \item Caiçaras
%   \item Riberinhos
%   \item \ldots
%   \end{itemize}

% \end{frame}

% }

% \begin{frame}

%   \frametitle{\emph{Digital divide} in Brasile}
%   Popolazione con accesso ad Internet in Brasile\footnote{Fonte: intervista al Segretario Esecutivo del Ministero delle
%     Comunicazioni, Cesar Alvarez, su dati IBGE 2009.}:
%   \begin{itemize}
%     \item $\sim$ 30\% della popolazione
%     \item $\sim$ 6\% della popolazione in area rurale  
%   \end{itemize}
%   \vfill
%   Come si affronta il digital divide?
%   \begin{itemize}
%     \item Alfabetizzazione informatica
%     \item Accesso a internet
%     \item \ldots
%     \item Ricerca e sviluppo?
%     \end{itemize}

%  \end{frame}

% \begin{frame}

%   \frametitle{Neutralità tecnologica}

%   \begin{quote}
%     ``Siamo portati a pensare al mezzo come neutrale, ma se prendiamo
%     ad esempio i principali e più diffusi mezzi di comunicazione, le
%     lingue, possiamo intuire come queste non siano interscambiabili
%     essendo l'espressione delle culture e delle società che le usano e
%     le vivono.''
%   \end{quote}
  
% \end{frame}


% \begin{frame}
%   \frametitle{Internet, autonomia e libertà tecnologica}
%   \framesubtitle{Dagli RFC\ldots}
%   \begin{quote}
%   ``Internet è nata dal confronto aperto tra i responsabili di diverse
%   reti, uniti dalla volontà di connettere le loro differenti
%   realtà. La nascita di nuovi servizi, per queste reti eterogenee, era
%   basata sulla discussione e il confronto.'' 
% \end{quote}

% \end{frame}

% \begin{frame}
%   \frametitle{Internet, autonomia e libertà tecnologica}
%   \framesubtitle{\ldots ai ToS/API/Cloud}
%   \begin{quote}
%   ``Negli ultimi anni la diffusione della banda larga, ma sopratutto
%   strategie come quella adottata da Google, hanno  trasformato il 
%   concetto stesso di internet che da rete globale di reti eterogenee,
%   diventa principalmente una rete per la globalizzazione di servizi
%   fortemente centralizzati e uniformati.'' 
%   \end{quote}
% \end{frame}

% \begin{frame}
%   \frametitle{Internet, autonomia e libertà tecnologica}
%   \framesubtitle{\ldots ai ToS/API/Cloud}
%   \begin{quote}
%     ``Secondo la società di consulenza Gartner, nel 2016, tutte le
%     compagnie contemplate dal \emph{Forbes Global 2000} faranno uso di
%     soluzioni \emph{cloud} \footnote{Tratto dall'articolo: ``Six global trends
%       shaping the business world'', del 2011, pubblicato sulla rivista EY
%       Insights di Ernst \& Young}.''
%   \end{quote}
% \end{frame}

% \begin{frame}
%   \frametitle{Dalle maestranze informatiche\ldots}
%   \framesubtitle{\ldots ai nuovi operai}
%   \begin{quotation}
%     ``\ldots un tempo lo sviluppo seguiva un modello \emph{bottom up}, per
%     cui le maestranze informatiche sviluppavano sistemi ad hoc per le
%     esigenze locali, per poi in seguito aprire una discussione in
%     rete, con i loro corrispettivi, per definire dei protocolli
%     standard e mettere in comunicazione il tutto.

%     Oggi si passa ad un modello di sviluppo \emph{top down}, per cui
%     nuovi servizi vengono lanciati basandosi su indagini di mercato e
%     test su campioni di utenti.''
%   \end{quotation}
% \end{frame}


% \section{Reti federate eventualmente connesse}

% \begin{frame}
%   \frametitle{La Rete Mocambos}
%   \framesubtitle{Mappa delle comunità}
% 	\begin{figure}
% 		\includegraphics[height=0.7\textheight]{./Figure/MappaRedeMocambos.pdf}
% 	\end{figure}
% \end{frame}

% \begin{frame}
%   \frametitle{Reti federate eventualmente connesse}
%   \begin{quote}
%     ``una rete federata basata su connessioni non sempre disponibili,
%     quali le connessioni satellitari, con l'esigenza di mantenere i
%     servizi federati attivi, anche in assenza di comunicazione. I
%     servizi federati sono inoltre ottimizzati per la resilienza del
%     sistema e la riduzione del traffico di rete esterno, attraverso
%     strategie di replicazione, sincronizzazione e memorizzazione dei
%     dati sull'infrastruttura logica/fisica locale.''
%     \end{quote}
% \end{frame}


% % Local background must be enclosed by curly braces for grouping.
% {
% %\usebackgroundtemplate{\includegraphics[width=.4\textwidth]{include/pista.jpg}}%
% \begin{frame}
%   \frametitle{Reti federate eventualmente connesse}
%   \framesubtitle{Tecnologie e strumenti utili}
%   Tecnologie libere (FLOSS) considerate:
%   \begin{itemize}
%     \item LDAP
%     \item XMPP
%     \item OpenID
%     \item OAuth
%     \item Django / Ruby On Rails
%     \item git / git-annex
%     \item rsync
%     \end{itemize}
% \end{frame}
% }


% \section{Un'architettura per la Rete Mocambos}

% \begin{frame}
%   \frametitle{Un'architettura per la Rete Mocambos}
%   \framesubtitle{Specifica dei requisiti}
%   \begin{itemize}
%     \item Identità di rete
%     \item Autenticazione decentrata
%     \item Sincronizzazione
%     \item Riproducibilità
%     \item Manutenzione
%     \item Sviluppo
%     \end{itemize}

% \end{frame}

% \begin{frame}
%   \frametitle{Un'architettura per la Rete Mocambos}
%   \framesubtitle{Strumenti e pratiche per lo sviluppo}
%   \begin{itemize}
%     \item Documentazione e ``versionamento'': wiki, git
%     \item Sistema operativo: Debian e Ubuntu
%     \item Linguaggi di programmazione: Python, shell script
%     \item Virtualizzazione: Virtualbox, Virtualenv
%     \end{itemize}

% \end{frame}

% \begin{frame}
%   \frametitle{Un'architettura per la Rete Mocambos}
%   \framesubtitle{Architettura di base}
% 	\begin{figure}
% 		\includegraphics[height=0.7\textheight]{./Figure/SchemaServer_ReteMocambos-crop.pdf}
% 	\end{figure}
% \end{frame}

% \section{Un prototipo di servizio federato}

% \begin{frame}
%   \frametitle{Un prototipo di servizio federato}
%   \framesubtitle{Sistema di pubblicazione e diffusione di contenuti multimediali}
% 	\begin{figure}
% 		\includegraphics[width=\textwidth]{./Figure/SequenceDiagram_NuovoOggetto-crop.pdf}
% 	\end{figure}
% \end{frame}

% \begin{frame}
%   \frametitle{Un prototipo di servizio federato}
%   \framesubtitle{Archivio multimediale}
% \emph{git-annex} supporta:
% \begin{itemize}
% \item localizzazione delle copie (\emph{location tracking})
% \item scaricamento selettivo dei contenuti
% \item gestione della fiducia dei \emph{repository}
% \item gestione del numero di repliche minimo
% \item vari \emph{backend} per le chiavi (SHA, WORM)
% \item vari \emph{backend} per i contenuti/valori (BUP, rsync, web, S3)
% \end{itemize}

% \end{frame}

% \section{Conclusioni e sviluppi futuri}

% \begin{frame}
%  \frametitle{Conclusioni e sviluppi futuri}
%   \begin{quote}
%     ``L'architettura e il prototipo sviluppati sono in corso di
%     implementazione, su scala ridotta, a partire dalle comunità più
%     strutturate e che fungono già da poli regionali, chiamati Nuclei
%     di Formazione Continua, che sono attualmente dieci,
%     geograficamente ben distribuiti sul territorio nazionale
%     brasiliano.''
%   \end{quote}
% \end{frame}

% \begin{frame}
%  \frametitle{Conclusioni e sviluppi futuri}
%   \framesubtitle{Risultati\ldots}
%   Funzionalità implementate:
%   \begin{itemize}
%   \item autenticazione LDAP (con gestione basica dei gruppi)
%   \item creazione e upload di contenuti audio, immagini e video
%   \item distribuzione tramite \emph{git-annex}
%   \item sincronizzazione degli oggetti sui portali django (ricreando
%     gli oggetti relativi ai contenuti distribuiti via
%     \emph{git-annex})
%   \end{itemize}
% \end{frame}
  
% \begin{frame}
%  \frametitle{Conclusioni e sviluppi futuri}
%   \framesubtitle{\ldots e prossime versioni}
 
%   Da implementare:
%   \begin{itemize}
%   \item trasferimento selettivo dei contenuti/valori basato sull'uso
%     statistico o su richiesta
%   \item sviluppo di in interfaccia di visualizzazione e pubblicazione
%     per l'utente finale
%   \item gestione del DIT dell'LDAP tramite script e/o portale
%   \end{itemize}
% \end{frame}

% \begin{frame}
%  \frametitle{Partner}
%  La Rete Mocambos conta con l'appoggio di altre reti, collettivi e istituzioni
  
%   \begin{itemize}
%   \item Pontos de Cultura 
%   \item Metareciclagem
%   \item Rete Dyne 
%   \item \ldots
%   \end{itemize}
  
%   Partner istituzionali:
%   \begin{itemize}
%   \item GESAC
%   \item Telecentros.BR
%   \item Secretaria de Políticas de Promoção da Igualdade Racial da
%     Presidência da República
%   \item Unicamp 
%   \item \ldots
%   \end{itemize}
  
% \end{frame}

% \section{Extra}

% \begin{frame}
%   \frametitle{Accordo con il Ministero delle Comunicazioni}
%   \framesubtitle{Video-documento}
% 	\begin{figure}
% 		\includegraphics[height=0.6\textheight]{./Figure/videomc.pdf}
% 	\end{figure}

%  \end{frame}

 \section{Asé}

 \begin{frame}
  \frametitle{Fine}
  \begin{center}
   \huge Obrigado pela atenção! :) \\
   \vfill
   \large  
   Antonio Carlos dos Santos / TC \\
   \normalsize
   tc@mocambos.net
   \vfill
   \begin{figure}[htb]
     \begin{minipage}[c]{0.10\textwidth}
       \includegraphics[width=\textwidth]{./FIG/NPDD.pdf}
  \end{minipage}
  \begin{minipage}[c]{0.60\textwidth}
    \footnotesize
    Nucleo de Pesquisa e Desenvolvimento Digital \\*
    Casa de Cultura Tainã/Rede Mocambos\\*
    http://wiki.mocambos.net/wiki/NPDD
  \end{minipage}

\end{figure}
\end{center}
\end{frame}

} % Fim do background


\end{document}
